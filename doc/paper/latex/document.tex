\documentclass[11pt, a4paper, twoside, titlepage]{article}
\usepackage{ graphicx}
\usepackage[colorlinks=true, 
            urlcolor=blue]{hyperref}
\usepackage{mathtools}
\usepackage[margin=1in]{geometry}
\usepackage{fixltx2e} 
\usepackage[T1]{fontenc}
\usepackage[utf8]{inputenc}
\usepackage{tikz}
\usepackage{centernot}
\usepackage{listings}
\usetikzlibrary{shadows}
\usepackage[toc,page]{appendix}
\usepackage{breakurl}
\newcommand*\keystroke[1]{%
  \tikz[baseline=(key.base)]
    \node[%
      draw, 
      fill=white,
      drop shadow={shadow xshift=0.25ex,shadow yshift=-0.25ex,fill=black,opacity=0.75},
      rectangle,
      rounded corners=2pt,
      inner sep=1pt,
      line width=0.5pt,
      font=\scriptsize\sffamily
    ](key) {#1\strut}
  ;
}

\newcommand{\todo}[1]{\textit{#1}}

\title{MSCViewer v1.0} \author{ Roberto Attias  \\
    NOSTG CRBU Advanced Tech \\
    CISCO\\
    }
\date{\today}  

\definecolor{dkgreen}{rgb}{0,0.6,0}
\definecolor{gray}{rgb}{0.5,0.5,0.5}
\definecolor{mauve}{rgb}{0.58,0,0.82}
\definecolor{lightcyan}{rgb}{0.83,0.97, 1}

\lstset{frame=tb,
  language=Python,
  aboveskip=3mm,
  belowskip=3mm,
  showstringspaces=false,
  columns=flexible,
  basicstyle={\small\ttfamily},
  numbers=left,
  numberstyle=\tiny\color{gray},
  keywordstyle=\color{blue},
  commentstyle=\color{mauve},
  stringstyle=\color{dkgreen},
  backgroundcolor=\color{lightcyan},
  frame=shadowbox,
  xleftmargin=2em,
  framexleftmargin=1.5em, 
  rulesepcolor=\color{gray},
  breaklines=true,
  breakatwhitespace=true
  tabsize=3
}

\begin{document}
% Hint: \title{what ever}, \author{who care} and \date{when ever} could stand 
% before or after the \begin{document} command 
% BUT the \maketitle command MUST come AFTER the \begin{document} command! 
\maketitle 

\section{Introduction}



Modern core routers are complex system where a large part of the functionalities
is implemented in software. From a single-process/single-threaded model
systems have evolved to multiple threads, multiple processes, and even multiple 
nodes operating cooperatively. Even when processes are single threaded, they 
often carry out asynchronous comunication and react to events, effectively
handling concurrent execution contexts within a single thread. Control plane,
admin plane and to some extent data plane implement multiple functionalities
through transactions and interactions across different processes and threads.
Some thread may cooperate to carry out multiple occurrencies of the same or
different functionality. Hence, verifying the correct execution of a
functionality is a complex task, typically requiring engineers to examine traces
produced by various processes. Due to the complexity of the system
very few engineers are familiar enough with the internals of all the
processes involved, and therefore able to confidently interpret all traces. Thus
triaging of bugs often requires multiple engineers attention for a considerable
amount of time, diverting those engineers from more productive tasks.
 
This reliance on engineers for doing even the first level of triaging is very
expensive and not scalable. In this paper we present a solution addressing this  
problem. The solution is based on emitting traces in a formal language where 
events and interactions are captured, and from which a model can be
automatically built. Through a second formal language we support specification
of expected flows for various functionalities. Finally, we provide a tool which
can  apply flow specifications to the model created from the traces, allowing
to do both visual inspection of flows in the form of sequence diagram charts as
well as GUI-less verification, for use in automatic tests. 
  

MSCViewer is a
\href{http://en.wikipedia.org/wiki/Message_Sequence_Chart#Live_Sequence_Charts}{message sequence chart}
visualization and analysis tool. The tool was created in NOSTG in the context of
the NG-XR admin-plane development, and it is part of a set of technologies known
under the name of RISE (Reusable Infrastructure Software Elements).
while MSCViewer is indepenent from other RISE elements, such elements can
be easily integrated in new or existing application to 
leverage the power of Message Sequence Chart analysis efficiently and with
minimal effort.
A brief overview of RISE will be provided in section~\ref{sec:RISE}. 

\include{model}
\include{modeling_language}
\include{mscviewer_gui}
\include{flow_definition}
\include{rise_overview}

\appendix
\section{Useful Links}
\begin{tabular}{l l p{10.0cm}}
CTRACE wiki  &
\url{http://twiki.cisco.com/Main/RiseCTRACE}
\\ \\
CAPI spec &
\url{http://wwwin-eng.cisco.com/Eng/OIBU/Panini/SW_Specs/Calvados_API_Spec.docx}
\\
\end{tabular}
\end{document}
